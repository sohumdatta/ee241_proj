\documentclass[10pt,journal,compsoc]{IEEEtran}

\ifCLASSINFOpdf
  \usepackage[pdftex]{graphicx}
  \graphicspath{{../pdf/}{../jpeg/}}
  \DeclareGraphicsExtensions{.pdf,.jpeg,.png}
\else
  \usepackage[dvips]{graphicx}
  \graphicspath{{../eps/}}
  \DeclareGraphicsExtensions{.eps}
\fi

\usepackage{amsmath}
% Note that the amsmath package sets \interdisplaylinepenalty to 10000
% thus preventing page breaks from occurring within multiline equations. Use:
\interdisplaylinepenalty=2500

% *** ALIGNMENT PACKAGES ***
%
%\usepackage{array}
% Frank Mittelbach's and David Carlisle's array.sty patches and improves
% the standard LaTeX2e array and tabular environments to provide better
% appearance and additional user controls. As the default LaTeX2e table
% generation code is lacking to the point of almost being broken with
% respect to the quality of the end results, all users are strongly
% advised to use an enhanced (at the very least that provided by array.sty)
% set of table tools. array.sty is already installed on most systems. The
% latest version and documentation can be obtained at:
% http://www.ctan.org/pkg/array


%\usepackage{mdwmath}
%\usepackage{mdwtab}
% Also highly recommended is Mark Wooding's extremely powerful MDW tools,
% especially mdwmath.sty and mdwtab.sty which are used to format equations
% and tables, respectively. The MDWtools set is already installed on most
% LaTeX systems. The lastest version and documentation is available at:
% http://www.ctan.org/pkg/mdwtools


% IEEEtran contains the IEEEeqnarray family of commands that can be used to
% generate multiline equations as well as matrices, tables, etc., of high
% quality.


%\usepackage{eqparbox}
% Also of notable interest is Scott Pakin's eqparbox package for creating
% (automatically sized) equal width boxes - aka "natural width parboxes".
% Available at:
% http://www.ctan.org/pkg/eqparbox




% *** SUBFIGURE PACKAGES ***
%\ifCLASSOPTIONcompsoc
%  \usepackage[caption=false,font=footnotesize,labelfont=sf,textfont=sf]{subfig}
%\else
%  \usepackage[caption=false,font=footnotesize]{subfig}
%\fi
% subfig.sty, written by Steven Douglas Cochran, is the modern replacement
% for subfigure.sty, the latter of which is no longer maintained and is
% incompatible with some LaTeX packages including fixltx2e. However,
% subfig.sty requires and automatically loads Axel Sommerfeldt's caption.sty
% which will override IEEEtran.cls' handling of captions and this will result
% in non-IEEE style figure/table captions. To prevent this problem, be sure
% and invoke subfig.sty's "caption=false" package option (available since
% subfig.sty version 1.3, 2005/06/28) as this is will preserve IEEEtran.cls
% handling of captions.
% Note that the Computer Society format requires a sans serif font rather
% than the serif font used in traditional IEEE formatting and thus the need
% to invoke different subfig.sty package options depending on whether
% compsoc mode has been enabled.
%
% The latest version and documentation of subfig.sty can be obtained at:
% http://www.ctan.org/pkg/subfig



% *** PDF, URL AND HYPERLINK PACKAGES ***
\usepackage{url}
% url.sty was written by Donald Arseneau. It provides better support for
% handling and breaking URLs. url.sty is already installed on most LaTeX
% systems. The latest version and documentation can be obtained at:
% http://www.ctan.org/pkg/url
% Basically, \url{my_url_here}.

\begin{document}

\appendices
% you can choose not to have a title for an appendix
% if you want by leaving the argument blank
\section{The C Gate and the Virtex-5 FPGA}
The Muller C gate is a sequential circuit element which switches only when all its inputs have
the same value.
When inputs disagree, the gate output retains its previous value.
Any multi-input C gate can be expressed as a cascade of 2-input C gates
(henceforth MULLER).
This appendix describes the MULLER gate and derives its realization on 
the Xilinx Virtex-5 FPGA.
\\

Boolean expressions of sequential elements need special notation to distinguish
between the current value and next value of signals. 
We denote both the signal and its current value by the signal name. 
Hence, $a$ denotes the value of signal 'a' at the current instant. 
Let $a'$ denote its value at the next instant and $\overline{a}$ its
logical negation. $a''$ will be the value of 'a' after $a'$.
Also, let $\oplus$, $\cdot$ and $+$ represent logical XOR, AND and OR respectively.
\\

The MULLER gate can be modelled as a latch which is transparent to one of its
inputs only when both of them agree:
\begin{equation} \label{eq:muller_c}
	\texttt{MULLER}(a,b)' = 
	\overline{(a \oplus b)} \cdot a\ +\ (a \oplus b) \cdot \texttt{MULLER}(a,b)
\end{equation}
%The C gate is the principal element of asynchronous circuits as it implements
%the indication principle -- it indicates a module to start working only when
%its inputs are ready. 
%\\

%----------------------------------------------------------------------
% Subsection: Virtex 5 primitives
\subsection{Xilinx Virtex-5 Primitives}

The Xilinx Virtex-5 FPGA contains 4 storage elements in each SLICE.
A storage element can be configured as a level-sensitive latch with input
driven by a LUT in the same SLICE. 
The latch is transparent when the control signal clock \texttt{CLK} is LOW. 
Any latch element can either be instantiated directly as a 
\textsl{Transparent Data Latch with Asynchronous Clear and Preset and Gate
Enable} (LDCPE) primitive or inferred by the Xilinx Synthesis Tool (XST).

Although XST enables optimization across modules, inferred synthesis may 
combine handshake stages to produce unwanted behaviour [TODO: experiment to
see].
For guaranteed functionality, the LDCPE primitive must be used. 
Its inputs are asynchronous clear (\texttt{CLR}) and preset
(\texttt{PRE}), gate enable (\texttt{GE}), gate (\texttt{G}) and data
(\texttt{D}):
\begin{equation}\label{eq:virtex_latch}
	\texttt{LDCPE}' = \overline{\texttt{CLR}} \cdot 
	(\texttt{PRE} + (\texttt{GE} \cdot \texttt{G}) \cdot \texttt{D}) +
	\overline{\texttt{CLR}} \cdot 
	(\overline{\texttt{GE} \cdot \texttt{G}}) \cdot \texttt{LDCPE}
\end{equation}

A straightforward way of synthesizing MULLER on Virtex-5 would be to program
a LUT for the XNOR gate $\overline{(a \oplus b)}$ and connect its output to
\texttt{CLK}.
%----------------------------------------------------------------------
% Subsection: C gate using RS and SR Latches
%----------------------------------------------------------------------

\subsection{C Gate using RS and SR Latches}

Although MULLER can be implemented by a single latch and LUT, it is prone
to spurious transitions in inputs. 
Using two latches reduces the risk of an output transition due to glitching.
A well known implementation of MULLER using RS and SR latches due to Murphy
[TODO: cite murphy] can be used for this purpose.

Both SR and RS latches have two inputs \texttt{set} (\texttt{S}) and
\texttt{reset} (\texttt{R}). 
SR latch is equivalent to the RS latch with \texttt{R} and \texttt{S} inputs
interchanged and the output inverted.
\begin{equation}\label{eq:rslatch}
	\texttt{RSLatch}(\texttt{S},\texttt{R})' 
	= \overline{\texttt{R}} \cdot (\texttt{S} + \texttt{RSLatch})
\end{equation}
\begin{equation}\label{eq:srlatch}
	\texttt{SRLatch}(\texttt{S},\texttt{R})' 
	= \texttt{S} + \overline{\texttt{R}} \cdot \texttt{SRLatch}
\end{equation}

For brevity $\texttt{MULLER}(a,b)$ is denoted by $c$. 
 One can easily simplify eqn. \ref{eq:muller_c} (using identity $x + \overline{x}
 \cdot y = x + y$ ) to obtain:
\begin{equation}\label{eq:muller_c_simplified}
\begin{split}
	\texttt{MULLER}(a,b)' &= c'\\
	&= (a \cdot b + \overline{a} \cdot \overline{b}) \cdot a 
		+ (a \cdot \overline{b} + \overline{a} \cdot b) \cdot c \\
	&= a \cdot (b + \overline{b} \cdot c) + \overline{a} \cdot b \cdot c \\
	&= a \cdot c + b \cdot (a + \overline{a} \cdot c) \\
	&= a \cdot b + (a + b) \cdot c
\end{split}
\end{equation}
To express MULLER in terms of SR and RS latches, 
observe that eqn. \ref{eq:muller_c_simplified} contains only true values of
inputs but \texttt{R} appears in both SR and RS latches (eqs. \ref{eq:srlatch}
and \ref{eq:rslatch}) as $\overline{R}$. 
Hence, \texttt{R} of the latches will be $\overline{a}$ or $\overline{b}$.
Only the expression for RS latch (eqn. \ref{eq:rslatch}) has a minterm containing both \texttt{R}
and \texttt{S}, which will produce the minterm $ a \cdot b $ in eqn. \ref{eq:muller_c_simplified}.
\\
Therefore, the RS latch appears in the first stage producing the minterms $a
\cdot b$ and $a \cdot c$ or $b \cdot c$ (depending on \texttt{R} being
$\overline{a}$ or $\overline{b}$ respectively). 
The SR latch forms the second stage.
Since \texttt{S} appears alone in its expression, we connect the RS output
to \texttt{S} and the input other than \texttt{R} of previous stage to the
\texttt{R} of the second stage.

Representing the RS output by $p$ and the subsequent SR output by $q$,
we have:
\begin{equation}\label{eq:muller_rs}
\begin{split}
	\texttt{RSLatch}(\texttt{S}=a, \texttt{R}=\overline{b}) &= p' = a \cdot b + b \cdot p \\
	\texttt{SRLatch}(\texttt{S}=p, \texttt{R}=\overline{a}) &= q' = p + a \cdot q \\
\end{split}
\end{equation}
From above, we have $q'' = p' + a' \cdot q' = (a \cdot b + b \cdot p) + 
a' \cdot (p + a \cdot q)$. 
Assuming input steady state i.e.\ $a'' = a' = a$ and $b'' = b' = b$,
\begin{equation}
\begin{split}
	q''& = a \cdot b + a \cdot p + b \cdot p + a \cdot q\\
	&= a \cdot b \cdot (1 + q) + (a + b ) \cdot p + a \cdot q\\
	&= a \cdot b + (a + b ) \cdot p + a \cdot q + a \cdot b \cdot q\\
	&= a \cdot b + (a + b ) \cdot p + (a + b) \cdot a \cdot q\\
	&= a \cdot b + (a + b ) \cdot (p + a \cdot q)\\
	&= a \cdot b + (a + b ) \cdot q'
\end{split}
\end{equation}

This proves the equivalence of the latch pair (eqn. \ref{eq:muller_rs}) to MULLER
(eqn. \ref{eq:muller_c_simplified}).

It may appear that the heuristic used to connect the pair just happened to
prove equivalent to MULLER. 
However, the circuit really was synthesized using reductions on a 
Signal Transition Graphs (STG) specification [TODO: cite murphy]. 
STGs are directed graphs with nodes corresponding to minterms in eqn.
\ref{eq:muller_c_simplified} [TODO: Find citation]. 
The synthesis algorithm does something similar to matching leaf nodes of the
latch STGs with that of MULLER -- like our heuristic [TODO: Find
citation]. 
Also, note that the difference of SR and RS latch expressions (eqs.
\ref{eq:srlatch} and \ref{eq:rslatch}) helped using the heuristic. 
Though an RS latch can be converted to SR latch easily, it would be much harder
to build the pair using 2 SR latches.

The LDCPE primitive can be easily configured as RS and SR latches. 
To set $\texttt{LDCPE} = \texttt{RSLatch}(\texttt{S}, \texttt{R})$ by comparing
eqs. \ref{eq:virtex_latch} and \ref{eq:rslatch}, we set
$\texttt{CLR} = \texttt{R}$ and $\overline{\texttt{GE} \cdot \texttt{G}} = 1$.
Then, $\texttt{PRE} = \texttt{S}$.
Similarly, for $\texttt{SRLatch}$ set $\overline{\texttt{CLR}} = 1,
 (\texttt{GE} \cdot \texttt{G}) = \texttt{R}, \texttt{D} = 0$ and $\texttt{PRE}
 = \texttt{S}$.


 
% Can use something like this to put references on a page
% by themselves when using endfloat and the captionsoff option.
\ifCLASSOPTIONcaptionsoff
  \newpage
\fi



% trigger a \newpage just before the given reference
% number - used to balance the columns on the last page
% adjust value as needed - may need to be readjusted if
% the document is modified later
%\IEEEtriggeratref{8}
% The "triggered" command can be changed if desired:
%\IEEEtriggercmd{\enlargethispage{-5in}}
% references section

% that's all folks
\end{document}


